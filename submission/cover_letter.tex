\documentclass[11pt]{letter}
\usepackage[margin=1in]{geometry}
\usepackage{hyperref}

\signature{Ian Todd\\Sydney Medical School\\University of Sydney}
\address{Ian Todd\\Sydney Medical School\\University of Sydney\\Sydney, NSW 2006, Australia\\itod2305@uni.sydney.edu.au}

\begin{document}

\begin{letter}{Editorial Office\\Synthese\\Springer Nature}

\opening{Dear Editors,}

I am pleased to submit the manuscript ``Oscillatory Incompleteness: Gödel, Symbol Formation, and High-Dimensional Dynamics'' for consideration in \textit{Synthese}.

This paper proves a Gödel-style incompleteness theorem for oscillatory dynamical systems that generate symbolic codes via coarse-grained observation. The technical result---that such systems inherit incompleteness when they can encode arithmetic---follows from established work on dynamical systems and computation (Moore 1990, Platzer 2012). Our contribution is interpretive: we argue that incompleteness arises specifically from the \emph{dimensional mismatch} between high-dimensional pre-symbolic dynamics and low-dimensional symbolic codes.

The paper connects several areas within Synthese's scope:

\begin{itemize}
    \item \textbf{Foundations of mathematics}: We provide a physical interpretation of Gödel's theorems, recasting incompleteness as a consequence of code formation rather than syntactic self-reference.

    \item \textbf{Philosophy of science}: The framework applies to any physical system that generates discrete states from continuous dynamics---including neural oscillations, chromosomal dynamics, and bioelectric signaling.

    \item \textbf{Formal methods}: We offer a categorical reformulation using coalgebras and topos-theoretic language, connecting to recent work on arithmetic universes.
\end{itemize}

The manuscript includes:
\begin{itemize}
    \item Formal definitions and three theorems with complete proofs (main text sketches, appendix details)
    \item Application to neural oscillatory dynamics
    \item Discussion of scope beyond neuroscience (cellular, ecological, cosmological)
    \item Simulation code demonstrating the core constructions (available on GitHub)
\end{itemize}

This work has not been published elsewhere and is not under consideration at another journal. The manuscript is approximately 7,500 words (16 pages double-spaced).

This paper is part of a broader research programme examining how living things---from cells to brains---make decisions by forming discrete symbolic representations from high-dimensional continuous dynamics. I believe it will interest Synthese's readership in philosophy of mathematics, philosophy of science, and formal methods, offering a novel physical grounding for one of logic's most celebrated results.

\closing{Sincerely,}

\end{letter}
\end{document}
